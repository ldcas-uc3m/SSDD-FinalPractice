\subsection{Servidor}

Para ejecutar el servidor hay que ejecutar el archivo Makefile que está incluido dentro de la carpeta \texttt{src}.\\

A la hora de compilar el servidor, es necesario compilar previamente los archivos de
librería: \texttt{linked\_list.c}, \texttt{server\_list.c}, y \texttt{lines.c}. También se incluye \texttt{log.c} el cual permite crear mensajes que se muestran por consola durante el modo debug para rastrear posibles errores.\\

Para compilar es tan simple como ejecutar:
\begin{lstlisting}
    $ cd src/
    $ make
\end{lstlisting}

La ejecución del servidor se hace después de la compilación, incluyendo una flag -p tras la que se indica el puerto que el servidor utilizará para la conexión.
\begin{lstlisting}
    $ ./servidor -p puerto
\end{lstlisting}