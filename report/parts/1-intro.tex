\section{Introducción y consideraciones generales}

La siguiente memoria relata las decisiones de diseño e implementación de la presente práctica, así como instrucciones de compilación y ejecución.\\

La funcionalidad de cada archivo fuente, de manera general, es:
\begin{itemize}
    \item \texttt{run.sh}: Ejemplo de uso y ejecución.
    \item \texttt{src/servidor.c}: Código fuente del servidor.
    \item \texttt{src/lib/linked\_list.c}: Librería de una lista enlazada para almacenar la información de cada cliente.
    \item \texttt{src/lib/lines.c}: Librería con funciones para la lectura y escritura de buffers de datos que se usarán para la comunicación del servidor.
    \item \texttt{src/lib/comm.h}: Archivo \textit{header} común donde se definen las variables comunes para la implementación del servidor.
    \item \texttt{src/lib/server\_impl.c}: Implementación de las funciones de comunicación del servidor con la estructura de datos
    \item \texttt{src/cliente.py}: Código fuente del cliente.
    \item \texttt{src/lib/lines.py}: Implementación de funciones para la lectura y escritura de buffers de datos que se usarán para la comunicación del cliente.
    \item \texttt{src/ws-format-service.py}: Código fuente del servicio web.
\end{itemize}