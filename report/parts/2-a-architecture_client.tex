\subsection{Cliente}

El código fuente del cliente fue desarrollado a partir del material de apoyo proporcionado con los profesores, el cual contaba con la implementación de la interfaz gráfica, así como un \textit{parser}.\\
Debido a que esas partes ya estaban implementadas, no las detallaremos aquí.\newline

Dado que la arquitectura del servidor es la de uno orientado a peticiones (usando \textit{sockets} TCP), cada una de las operaciones del cliente (halladas en la interfaz) consta de el mismo esquema general:
\begin{enumerate}
    \item Conectarse a la IP y puerto del servidor (proporcionados como argumentos en la línea de comandos) a través de un \textit{socket}, implementado en el método \texttt{client.socket\_connect()}.
    \item Opcionalmente, realizar alguna acción requerida por el protocolo.
    \item Enviar la petición al servidor a través del socket.
    \item Esperar la respuesta del servidor.
    \item Analizar la respuesta y mostrar el resultado en la interfaz.
    \item Cerrar el socket (cerrar la conexión).
\end{enumerate}

Las implementaciones de las operaciones del cliente consisten en seguir el protocolo proporcionado en el enunciado, por lo que no voy a detallar su diseño.\\
Sin embargo, algunas operaciones sí que requirieron de ciertas decisiones de diseño:
\begin{itemize}
    \item La operación \texttt{CONNECT} (método \texttt{client.connect()}) inicia un socket de escucha en un puerto libre, e inicia un hilo (método \texttt{client.listen()}) encargado de aceptar conexiones a ese puerto, recibir las peticiones pertinentes (\texttt{SEND MESSAGE} y \texttt{SEND MESS ACK}), y mostrarlas en la interfaz.
    \item La operación \texttt{DISCONNECT} (método \texttt{client.disconnect()}) cierra el puerto de escucha del cliente, generando una excepción en el hilo, la cual es capturada y hace que termine el hilo.
    \item La operación \texttt{SEND MESSAGE} (método \texttt{client.send()}) lanza una petición al servicio web de formateo de mensajes (método \texttt{client.format\_message()}, haciendo uso del módulo \href{https://pypi.org/project/zeep/}{zeep}) antes de enviar la petición. La IP y puerto del servicio son proporcionados como argumentos en la línea de comandos.
\end{itemize}

Durante toda la implementación del cliente se han usado bloques \texttt{try}-\texttt{except} para capturar posibles errores (conexión, etc), y mostrar un simple error en la interfaz en lugar de hacer que pare la ejecución.\\

También se decidió que, al cerrar la interfaz del cliente se envía al servidor un \texttt{DISCONNECT} y un \texttt{UNREGISTER}.